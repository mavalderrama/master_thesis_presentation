%%%%%%%%%%%%%%%%%%%%%%%%%%%%%%%%%%%%%%%%%
% Beamer Presentation
% LaTeX Template
% Version 1.0 (10/11/12)
%
% This template has been downloaded from:
% http://www.LaTeXTemplates.com
%
% License:
% CC BY-NC-SA 3.0 (http://creativecommons.org/licenses/by-nc-sa/3.0/)
%
%%%%%%%%%%%%%%%%%%%%%%%%%%%%%%%%%%%%%%%%%

%----------------------------------------------------------------------------------------
%	PACKAGES AND THEMES
%----------------------------------------------------------------------------------------

\documentclass{beamer}

\mode<presentation> {

% The Beamer class comes with a number of default slide themes
% which change the colors and layouts of slides. Below this is a list
% of all the themes, uncomment each in turn to see what they look like.

%\usetheme{default}
%\usetheme{AnnArbor}
%\usetheme{Antibes}
%\usetheme{Bergen}
%\usetheme{Berkeley}
%\usetheme{Berlin}
%\usetheme{Boadilla}
%\usetheme{CambridgeUS}
%\usetheme{Copenhagen}
%\usetheme{Darmstadt}
%\usetheme{Dresden}
%\usetheme{Frankfurt}
%\usetheme{Goettingen}
%\usetheme{Hannover}
%\usetheme{Ilmenau}
%\usetheme{JuanLesPins}
%\usetheme{Luebeck}
\usetheme{Madrid}
%\usetheme{Malmoe}
%\usetheme{Marburg}
%\usetheme{Montpellier}
%\usetheme{PaloAlto}
%\usetheme{Pittsburgh}
%\usetheme{Rochester}
%\usetheme{Singapore}
%\usetheme{Szeged}
%\usetheme{Warsaw}

% As well as themes, the Beamer class has a number of color themes
% for any slide theme. Uncomment each of these in turn to see how it
% changes the colors of your current slide theme.

%\usecolortheme{albatross}
%\usecolortheme{beaver}
%\usecolortheme{beetle}
%\usecolortheme{crane}
%\usecolortheme{dolphin}
%\usecolortheme{dove}
%\usecolortheme{fly}
%\usecolortheme{lily}
%\usecolortheme{orchid}
%\usecolortheme{rose}
%\usecolortheme{seagull}
%\usecolortheme{seahorse}
%\usecolortheme{whale}
%\usecolortheme{wolverine}

%\setbeamertemplate{footline} % To remove the footer line in all slides uncomment this line
%\setbeamertemplate{footline}[page number] % To replace the footer line in all slides with a simple slide count uncomment this line

%\setbeamertemplate{navigation symbols}{} % To remove the navigation symbols from the bottom of all slides uncomment this line
}

\usepackage{graphicx} % Allows including images
\usepackage{booktabs} % Allows the use of \toprule, \midrule and \bottomrule in tables
\usepackage{ragged2e}
\usepackage{etoolbox}
\usepackage[spanish]{babel}
\usepackage[utf8]{inputenc}

%----------------------------------------------------------------------------------------
%	TITLE PAGE
%----------------------------------------------------------------------------------------

\title[]{Computación de Borde con FPGA para IoT } % The short title appears at the bottom of every slide, the full title is only on the title page

\author{Manuel Alejandro Valderrama Florez} % Your name
\institute[PUJ] % Your institution as it will appear on the bottom of every slide, may be shorthand to save space
{
Pontificia Universidad Javeriana Cali \\ % Your institution for the title page
\medskip
\textit{mavalderrama@javerianacali.edu.co} % Your email address
}
\date{\today} % Date, can be changed to a custom date
\apptocmd{\frame}{}{\justifying}{}
\begin{document}

\begin{frame}
\titlepage % Print the title page as the first slide
\end{frame}

\begin{frame}
\frametitle{Agenda} % Table of contents slide, comment this block out to remove it
\tableofcontents % Throughout your presentation, if you choose to use \section{} and \subsection{} commands, these will automatically be printed on this slide as an overview of your presentation
\end{frame}

%----------------------------------------------------------------------------------------
%	PRESENTATION SLIDES
%----------------------------------------------------------------------------------------

%------------------------------------------------

\begin{frame}
\frametitle{MOTIVACIÓN}
\section{Motivación}

\end{frame}

%------------------------------------------------
\begin{frame}
\frametitle{OBJETIVOS}
\section{Objetivos}
Proponer una plataforma de computación de borde para IoT que proporcione aceleración por hardware y por medio de la  a través de la reconfiguración parcial, adicione funcionalidad de manera dinámica. 
\begin{enumerate}
    \item Definición de la plataforma
    \item Desarrollo de módulo de hardware como IP
    \item Integración e implementación de los módulos de hardware a la plataforma
    \item Comparar rendimiento Hardware vs Software
\end{enumerate}
    
\end{frame}
%------------------------------------------------


\begin{frame}
\frametitle{JUSTIFICACIÓN}

\end{frame}

%------------------------------------------------
\begin{frame}
\frametitle{ALCANCES}

\end{frame}
%------------------------------------------------
\begin{frame}
\frametitle{RECURSOS}
\textbf{Hardware}
\begin{itemize}
\item \href{https://store.digilentinc.com/zybo-zynq-7000-arm-fpga-soc-trainer-board/}{Tarjeta de Desarrollo ZYBO Zynq-7000 ARM/FPGA SoC}:
  \begin{itemize}
  \item Xilinx Zynq-7000 AP SoC XC7Z010-1CLG400C
  \item Dual-core ARM Cortex™-A9 
  \item 512 MB DDR3 
  \item 128 MB Quad-SPI Flash 
  \item 4 GB SD card 
  \item Onboard USB-JTAG Programming
  \item 10/100/1000 Ethernet 
  \item USB OTG 2.0 and USB-UART 
  \item Analog Devices ADAU1761 SigmaDSP® Stereo, Low Power, 96 kHz, 24-Bit Audio Codec
  \item Analog Devices ADV7511 High Performance 225 MHz HDMI Transmitter (1080p HDMI, 8-bit VGA, 128x32 OLED)
  \item PS \& PL I/O expansion (FMC, Pmod, XADC) 
  \end{itemize}
\end{itemize}
\end{frame}
%------------------------------------------------
\begin{frame}
\frametitle{RECURSOS}
    \textbf{Software}
    \begin{itemize}
    \item \href{http://www.xilinx.com/products/design-tools/vivado.html}{Vivado Design Suite 2017.4} \citep{Vivad98}: Programa para síntesis e implementación de hardware a partir de una descripción usando VHDL o Verilog.
    \item \href{http://www.xilinx.com/products/design-tools/vivado/integration/esl-design.html}{Vivado HLS 2017.4} \citep{Vivad77}: Programa para síntesis de alto nivel a un lenguaje de descripción de hardware (VHDL o Verilog).
    \item \href{http://www.xilinx.com/products/design-tools/embedded-software/petalinux-sdk.html}{Petalinux SDK 17.4} \citep{PetaL29}.
    \item \href{http://www.putty.org}{PuTTY}
    \end{itemize}
\end{frame}
%------------------------------------------------
\begin{frame}
\frametitle{METODOLOGÍA}

\end{frame}

%------------------------------------------------
\begin{frame}
\frametitle{ELECCIÓN PLATAFORMA}

\end{frame}

%------------------------------------------------
\begin{frame}
\frametitle{DISEÑO}

\end{frame}

%------------------------------------------------
\begin{frame}
\frametitle{VALIDACIÓN DEL MODELO}

\end{frame}

%------------------------------------------------
\begin{frame}
\frametitle{RESULTADOS}

\end{frame}

%------------------------------------------------
\begin{frame}
\frametitle{CONCLUSIONES}

\end{frame}

%------------------------------------------------

%------------------------------------------------

\begin{frame}[fragile] % Need to use the fragile option when verbatim is used in the slide
\frametitle{Citation}
An example of the \verb|\cite| command to cite within the presentation:\\~

This statement requires citation \cite{p1}.
\end{frame}

%------------------------------------------------

\begin{frame}
\frametitle{References}
\footnotesize{
\begin{thebibliography}{99} % Beamer does not support BibTeX so references must be inserted manually as below
\bibitem[Smith, 2012]{p1} John Smith (2012)
\newblock Title of the publication
\newblock \emph{Journal Name} 12(3), 45 -- 678.
\end{thebibliography}
}
\end{frame}

%------------------------------------------------

\begin{frame}
\Huge{\centerline{The End}}
\end{frame}

%----------------------------------------------------------------------------------------

\end{document}